\documentclass[11pt,twocolumn,a4paper]{article}

\usepackage{hyperref}
\usepackage{tikz}
\usepackage[cm]{fullpage}
\usepackage{amsmath,amssymb}
\newcommand{\RR}{\mathbb R}

\begin{document}

\begin{center}
  \textbf{Accelerated Failure Time models\\(parametric survival regression)}

toby@sg.cs.titech.ac.jp \\\textbf{Toby Dylan Hocking} \\30 Sept 2013
\end{center}

\thispagestyle{empty}

This is a class of models for

\begin{itemize}
\item several inputs $x_i\in\RR^p$,
\item using incomplete information about the outputs $y_i\in\RR^+$,
  meaning that sometimes we know only an interval of possible values
  $y_i\in(\underline y_i, \overline y_i)$,
\item assuming a distribution for the outputs such as
  $y_i\sim \text{LogNormal}(\mu_i, 1)$ or $y_i\sim
  \text{LogLogistic}(\alpha_i, 2)$ in the examples plotted below,
\item where the real-valued scale parameter $\mu_i= w' x_i$ or
  $\log\alpha_i=w'x_i$ is assumed to be a linear combination of the
  inputs,
\item and the center (mean or median) of the distribution is used as a
  predicted value $\hat y = f(x)\in\RR^+$.
\end{itemize}

Maximum likelihood inference is then performed on the variable weights
$w$. The only trick is that there is a non-standard likelihood
function, since we have incomplete information for the outputs $y_i$:

\begin{itemize}
\item If we observe the complete output value $y_i$, as in the right
  panel in the plot below, then the likelihood function is the
  \textbf{density} $d(y_i, \hat y)$, e.g. for the LogNormal.$\sigma$
  distribution,
  \begin{equation}
    \label{eq:lognormal}
    d(y_i, \hat y) = 
    \frac{1}{\hat y\sigma\sqrt{2\pi}}\exp\left(
      \frac{(\log y_i - \log \hat y)^2}{-2\sigma^2}
    \right)
  \end{equation}
  (more functions on the back for those who are interested)
\item If we observe only an interval $y_i\in(\underline y_i, \overline
  y_i)$, as in the left panel in the plot below, then the likelihood
  function is the \textbf{cumulative distribution function}
  $\int_{\underline y_i}^{\overline y_i} d(y, \hat y)\, dy$, which is
  often available in closed form (it is not closed-form for the
  LogNormal, but we can still evaluate it numerically).
\end{itemize}

\input{figure-loss}

%\newpage

These models can be used for predicting
\begin{itemize}
\item Survival times of patients $i$ that are treated for the same
  disease. Inputs $x_i$ are age at diagnosis, blood pressure, tumor
  size, etc. Output $y_i$ is the survival time after treatment. There
  are two cases at the time of the statistical analysis: (a) patient
  $i$ has died, so we observe $y_i$, or (b) patient $i$ is still
  alive, so we observe $y_i\in(t_i,\infty)$ where $t_i$ is the time
  patient $i$ has survived so far.
\item Penalties for related segmentation problems $i$, such as
  $\lambda_i\in\RR^+$ in $\min_z \text{Cost}_i(z) + \lambda_i\
  \text{Penalty}(z)$. Weak/incomplete labels about breakpoint
  locations do not indicate any single output $\lambda_i$ value, but
  instead provide an interval
  $\lambda_i\in(\underline\lambda_i,\overline \lambda_i)$ of optimal output
  values, and I used features $x_i$ of the segmentation problem such
  as number of points to segment and estimated variance. For more info
  see my ICML 2013 paper ``Learning Sparse Penalties for Change-point
  Detection...''
\end{itemize}

\textbf{Every concave maximum likelihood problem can also be thought
  of as a minimization of some convex loss function.} In machine
learning we focus on getting accurate predictions $\hat y$, so in the
plot below we show the loss function in terms of the predicted values
$L_i(\hat y)$. Note that there is a nuisance/shape parameter, such as
the standard deviation $\sigma$ in (\ref{eq:lognormal}), which affects
the shape of the plotted loss function but does not affect the
stationary point of the optimization with respect to $w$.

\textbf{Which distribution is better, LogNormal or LogLogistic?} There
are other distributions (Weibull, any other positive-valued
distribution, see back of this page), so how to choose which one is
best? It depends on your data, so first define an evaluation metric
such as the zero-one loss, and then use the distribution which gives
the lowest prediction error on a held-out test set of data.

\textbf{Free/open-source implementation}: \texttt{survreg()} function
in R package \texttt{survival}.

\newpage

\begin{table*}[t!]

  In survival analysis we have data $(t_i,x_i,\delta_i)$ for a set of
  patients $i\in\{1,\dots,n\}$. Like in usual regression, $x_i\in\RR^m$
  is a vector of input variables. However, we observe a time $t_i\in\RR^+$ and
  $\delta_i\in\{1=\text{death},0=\text{censor}\}$ which are related to the actual survival time $y_i\in\RR^+$ as
  follows:
\begin{equation}
  \label{eq:t_i}
  t_i =
  \begin{cases}
    \text{the amount of time patient $i$ lived} 
    & \Rightarrow y_i = t_i \text{ if
    }\delta_i=1=\text{death}\\
    \text{the amount of time until the study ended}
&\Rightarrow y_i > t_i  \text{ if }\delta_i=0=\text{censor}.
  \end{cases}
\end{equation}
So if $\delta_i=0=\text{censor}$, all we know is that the survival of
patient $i$ is at least $t_i$. So the likelihood can be written as
\begin{equation}
  \label{eq:surv_lik}
\prod_{\delta_i=1=\text{death}} d(t_i) \prod_{\delta_i=0=\text{censor}} s(t_i) =
\prod_{i=1}^n s(t_i) h(t_i)^{\delta_i}
\end{equation}
where $d$ is the density function, $s(t)=1-F(t)$ is the survival function,
$F$ is the cumulative distribution function, and $h$ 
is the hazard function ($h=d/s$).
Thus the log likelihood is
\begin{equation*}
\sum_{i=1}^n\underbrace{\log s(t_i) + \delta_i \log h(t_i) }_{\log\text{lik}_i}
\end{equation*}

\begin{center}
\begin{tabular}{ccc}
  Distribution & $-\log\text{lik}_i$ & Link \\
  \hline
  Exponential($\lambda_i$) & 
  $\lambda_i t_i - \delta_i \log \lambda_i$ &
  $\lambda_i = \exp w'x_i$\\
  Log-Logistic($\alpha_i, \beta$) &
  $(1+\delta_i)\log\left[1+ (t_i/\alpha_i)^\beta\right]
  -\delta_i\log(\beta t^{\beta-1} \alpha_i^{-\beta})$ &
  $\alpha_i = \exp w'x_i$\\
  Weibull($\gamma_i,k$) &
  $(t_i/\gamma_i)^k 
  -\delta_i\left[
    \log(k/\gamma_i) + (k-1)\log(t_i/\gamma_i)
  \right]$ &
  $\gamma_i = \exp w'x_i$\\
  Log-Normal($\mu_i, \sigma$) & 
  $-\log s(t_i)=1/2+\operatorname{erf}\left[
    \log(t_i/\mu_i)/(\sigma\sqrt{2})
  \right]$ &
  $\mu_i = w'x_i$
\end{tabular}
\end{center}

\begin{center}
\begin{tabular}{cllcc}
  Distribution & Prediction & $\hat t_i$ & 
  Surrogate loss $L(\delta,t,\hat t)$ \\
  \hline
  Exp($\lambda_i$) & Mean & $\lambda_i^{-1}$ &
  $t/\hat t + \delta\log\hat t$ \\
  Log-Logistic($\alpha_i, \beta$) & Median&  $\alpha_i$ &
  $(1+\delta)\log\left[1+ (t/\hat t)^\beta\right]
  -\delta\log(\beta t^{\beta-1} \hat t^{-\beta})$ \\
  Weibull($\gamma_i,k$) & Mean &  $\gamma_i$ &
  $(t/\hat t)^k - \delta\log(\hat t^{-1} k t^{k-1})$ \\
  Log-normal($\mu_i, \sigma$) &
  Mean & $\exp \mu_i$ & 
  $-\log s(t)=1/2+\operatorname{erf}\left[
    \log(t/\hat t)/(\sigma\sqrt{2})
  \right]$ 
\end{tabular}
\end{center}

\textbf{Exercise for the reader:}
consider the more general case of
real-valued interval output data, 
e.g. $y_1=(-\infty, 4)$, $y_2=(-3, \infty)$, $y_3=(-1, 2)$, $y_4=(5, 5)$.
Instead of using one of these positive-valued distributions,
consider an equivalent real-valued distribution
(e.g. logistic rather than log-logistic).
Write the maximum likelihood problem,
derive the equivalent convex minimization problem,
and write the surrogate loss $L(\underline y_i, \overline y_i, \hat t)$.
What is the gradient of the surrogate loss with respect to the 
optimization variables $w$?

\end{table*}

\end{document}
